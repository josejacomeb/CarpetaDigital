\documentclass[10pt,a4paper]{article}
\usepackage[utf8]{inputenc}
\usepackage{amsmath}
\usepackage{amsfonts}
\usepackage{amssymb}
\usepackage{graphicx}
\usepackage{fourier}
\author{José Jácome}
\begin{document}
%\chapter{Capitulo VII}
\section{TEORÍA DE SINGULARIDADES Y DEL RESIDUO.}
\subsection{SINGULARIDAD.-}
Un punto $z_0$ es un punto singular o una singularidad deu na función $F$, si $F$ es analítica en algún punto de toda variedad de $z_0$, excepto en $z_0$ mismo. \\
Existen Varios tipos de Singularidades.
\begin{enumerate}
\item[1º] \textbf{SINGULARIDAD AISLADA.-} El punto $z = z_0$ si $\exists \; \delta > 0$, tal que el círculo $\parallel z-z_0 \parallel= \delta$ no encierra puntos singulares distintos de $z_0$ (es decir $\exists \; V_{\delta} (z_0)$ sin singularidad). \\
Si tal $\delta \; \nexists $, decimos que $z_0$ es una singularidad no aislada. \\
Si $z_0$ no es un punto singular y si $\exists \; \delta > 0 \; / \; \parallel z - z_0 \parallel = \delta$ no encierra puntos singulares, decimos que $z_0$ es un punto ordinario de $F(z)$.
\item[2º] \textbf{POLOS.-} Si podemos encontrar un entero positivo $n$ tal que $\displaystyle{\lim_{ z \to z_0} (z-z_0)^n}$   \\$F(z) = A \neq 0$, entonces $z = z_0$ es llamado polo de orden $n$, si $n = 1$. $z_0$ es llamado un polo simple.\\
\textbf{Ejemplo.-} $f(z) = \dfrac{1}{(z-2)^3}$, se tiene un polo de orden tres en $z = 2$. \\
\textbf{Ejemplo.-} $f(z) = \dfrac{3 z - 2}{(z-1)^2(z+1)(z-4)}$;tiene un polo de orden dos en $z = 1$ y polos simples en $z = -1$ y $z = 4$ 

\end{enumerate}
\end{document}