\documentclass[10pt,a4paper]{article}
\usepackage[utf8]{inputenc}
\usepackage{amsmath}
\usepackage{amsfonts}
\usepackage{amssymb}
\usepackage{graphicx}
\usepackage{pgf,tikz}
\usetikzlibrary{arrows}
\pagestyle{empty}
\author{José Jácome}
\title{Corrección del Exámen III}
\begin{document}
\begin{center}
UNIVERSIDAD DE LAS FUERZAS ARMADAS ESPE\\
EXTENSIÓN LATACUNGA
\end{center}
MATEMÁTICA SUPERIOR\\
\textbf{1) Resuelva la siguiente EDP} 

$\frac { { d }^{ 2 }u\quad  }{ d{ x }^{ 2 } } =\frac { 4du }{ dy } \\
u=xy\\ { u }_{ xx }={ x }^{ " }y\quad \qquad \qquad { u }_{ y }=x{ y }^{ ' }\\ entonces\quad \\ { x }^{ " }y=4x{ y }^{ ' }\\ \\ \frac { { x }^{ " } }{ x } =\frac { 4{ y }^{ ' } }{ y } =-\lambda \\ \\ cuando\quad \lambda =0.\\ \\ { x }^{ " }=0\quad \quad \quad \quad \quad \quad \quad \quad \quad \quad \quad \quad \quad \quad \quad \quad \quad \qquad 4{ y }^{ ' }=0\\ { D }^{ 2 }=0\qquad \qquad \qquad \qquad \qquad \qquad \qquad \quad 4D=0\\ \\ x={ c }_{ 1 }{ e }^{ 0x }+{ c }_{ 2 }{ xe }^{ 0x }\qquad \qquad \qquad \qquad \qquad y={ c }_{ 3 }{ e }^{ 0x }={ c }_{ 3 }\\ x={ c }_{ 1 }+{ c }_{ 2 }x\\ \qquad \qquad u=xy=[{ c }_{ 1 }+{ c }_{ 2 }x]*{ c }_{ 1 }={ c }_{ 4 }+{ c }_{ 5 }x\\\\
cuando\quad \lambda ={ \alpha  }^{ 2 }\\ { x }^{ '' }+{ \alpha  }^{ 2 }x=0\qquad \qquad \qquad \qquad \qquad 4y^{ ' }+{ \alpha  }^{ 2 }y=0\\ D^{ 2 }+{ \alpha  }^{ 2 }=0\qquad \qquad \qquad \qquad \qquad 4D+{ \alpha  }^{ 2 }=0\\ D^{ 2 }=\pm j\alpha \qquad \qquad \qquad \qquad \qquad \qquad D=-\frac { { \alpha  }^{ 2 } }{ 4 } \\ x={ e }^{ 0x }[{ c }_{ 1 }cos\alpha x+{ c }_{ 2 }sen\alpha x]\\ x={ c }_{ 1 }cos\alpha x+{ c }_{ 2 }sen\alpha x\\ y={ c }_{ 3 }{ e }^{ -\frac { { \alpha  }^{ 2 } }{ 4 } x }\\ u=xy=[{ c }_{ 1 }cos\alpha x+{ c }_{ 2 }sen\alpha x]*{ c }_{ 3 }{ e }^{ -\frac { { \alpha  }^{ 2 } }{ 4 } x }\\\\
cuando\quad \lambda =-{ \alpha  }^{ 2 }\\ { x }^{ '' }-{ \alpha  }^{ 2 }x=0\qquad \qquad \qquad \qquad \qquad 4y^{ ' }-{ \alpha  }^{ 2 }y=0\\ D^{ 2 }-{ \alpha  }^{ 2 }=0\qquad \qquad \qquad \qquad \qquad 4D-{ \alpha  }^{ 2 }=0\\ D=\pm \alpha \qquad \qquad \qquad \qquad \qquad \qquad D=\frac { { \alpha  }^{ 2 } }{ 4 } \\ x={ c }_{ 1 }{ e }^{ \alpha x }+{ { c }_{ 2 }e }^{ -\alpha x }\\ x={ c }_{ 1 }cos\alpha x+{ c }_{ 2 }sen\alpha x\\ y={ c }_{ 3 }{ e }^{ \frac { { \alpha  }^{ 2 } }{ 4 } x }\\ u=xy=[{ c }_{ 1 }cos\alpha x+{ c }_{ 2 }sen\alpha x]*{ c }_{ 3 }{ e }^{ \frac { { \alpha  }^{ 2 } }{ 4 } x }
$\\\\\

\textbf{2) Halle la Serie de Fourier, la Transformada de Fourier, la transformada Inversa de Fourier y el Espectro de Frecuencias de la Función dada:} 

\[
f(t)= \left\{ \begin{array}{lcl}
0 & \mbox{ si } & -2 < t < 0 \\
& & \\
t & \mbox{ si } & 0 < t < 1\\
& & \\
1 & \mbox{ si } & 1 < t < 2\\
\end{array}
\right.
\]

%Codigo de la Imagen en Tikz de Geogebra
\definecolor{uuuuuu}{rgb}{0.26666666666666666,0.26666666666666666,0.26666666666666666}
\definecolor{qqffqq}{rgb}{0.0,1.0,0.0}
\begin{tikzpicture}[line cap=round,line join=round,>=triangle 45,x=2.0cm,y=2.0cm]
\draw[->,color=black] (-2.7662337551415663,0.0) -- (4.045820382626854,0.0);
\foreach \x in {-2.5,-2.0,-1.5,-1.0,-0.5,0.5,1.0,1.5,2.0,2.5,3.0,3.5,4.0}
\draw[shift={(\x,0)},color=black] (0pt,2pt) -- (0pt,-2pt) node[below] {\footnotesize $\x$};
\draw[->,color=black] (0.0,-0.1) -- (0.0,1.950875204839679);
\foreach \y in {,0.5,1.0,1.5}
\draw[shift={(0,\y)},color=black] (2pt,0pt) -- (-2pt,0pt) node[left] {\footnotesize $\y$};
\draw[color=black] (0pt,-10pt) node[right] {\footnotesize $0$};
\clip(-2.7662337551415663,-0.1) rectangle (4.045820382626854,1.950875204839679);
\draw [line width=2.0pt,color=qqffqq] (-2.0,0.0)-- (0.0,-0.0);
\draw [line width=2.0pt,color=qqffqq] (0.0,-0.0)-- (1.0,1.0);
\draw [line width=2.0pt,color=qqffqq] (1.0,1.0)-- (2.0,1.0);
\draw (-1.2667756845795235,0.33049309632908164) node[anchor=north west] {f(t) = 0};
\draw (0.14400475815358113,0.7577580304139655) node[anchor=north west] {f(t) = t};
\draw (1.4983539831773616,1.297885399917498) node[anchor=north west] {f(t) = 1};
\draw [dash pattern=on 1pt off 1pt,color=uuuuuu] (1.0,1.0)-- (1.0,0.0);
\draw [dash pattern=on 1pt off 1pt,color=uuuuuu] (2.0,1.0)-- (2.0,0.0);
\begin{scriptsize}
\draw[color=qqffqq] (-0.9765579935029991,-0.05646382510628491) node {$a$};
\draw[color=qqffqq] (0.6115777048879816,0.47560194186734406) node {$b$};
\draw[color=qqffqq] (1.522538790767072,0.9431748886017453) node {$c$};
\draw[color=uuuuuu] (0.8937337934346026,0.5723411722261857) node {$d$};
\draw[color=uuuuuu] (1.893372507142631,0.5723411722261857) node {$e$};
\end{scriptsize}
\end{tikzpicture}

\begin{center}
Resolución\\
Serie de Fourier\\
Término $a_0$\\
$T=4$ y $\delta = -2$
\end{center}

\begin{center}
$\displaystyle{a_0 = \dfrac{2}{T} \int_{\delta}^{\delta + T} f(t) dt = \dfrac{2}{4} \int_{-2}^{2} f(t) dt = \dfrac{1}{2} (0 + \int_{0}^{1} t dt + \int_{1}^{2} 1 dt)}$\\
$\displaystyle{a_0 = \dfrac{1}{2} (0 + \dfrac{t^2}{2} \mid^1_0  +t \mid^2_1 ) = \dfrac{1}{2} ( \dfrac{1^2 - 0}{2} +2-1)  = \dfrac{3}{4}}$\\
Término $a_n$\\
$\displaystyle{a_n = \dfrac{2}{T} \int_{\delta}^{\delta + T} f(t) cos (n \omega t)dt = \dfrac{2}{4} \int_{-2}^{2} f(t) cos (n \omega t)dt = \dfrac{1}{2} (0 + \int_{0}^{1} t cos (n \omega t) dt + \int_{1}^{2} cos (n \omega t) dt)}$\\
$\displaystyle{a_n = \dfrac{1}{2} [\dfrac{sen(n \omega t) t}{n \omega} - \int_{0}^{1} \dfrac{sen(n \omega t)}{n \omega} dt + \mid_{0}^{1} \dfrac{sen(n \omega) t}{n \omega} \mid_{1}^{2}]}$\\
$\displaystyle{a_n = \dfrac{1}{2} [(\dfrac{sen(n \omega t) }{n \omega} - \dfrac{cos(n \omega t)}{(n \omega)^2}   )\mid_{0}^{1} +  \dfrac{sen(n \omega t) t}{n \omega} \mid_{1}^{2}]}$\\
$\displaystyle{a_n = \dfrac{1}{2} [\dfrac{2 sen(n \omega) - 0 }{n \omega} - \dfrac{cos(n \omega) - cos(0)}{(n \omega)^2} + \dfrac{sen(2 n \omega ) -sen(n \omega) }{n \omega} ]}$\\
$\displaystyle{a_n = \dfrac{1}{2} [\dfrac{sen(2 n \omega)}{n \omega} - \dfrac{cos(n \omega) - 1}{(n \omega)^2}]}$\\
$\omega = \dfrac{2 \pi}{T} = \dfrac{2 \pi}{4} = \dfrac{\pi}{2} $\\
$\displaystyle{a_n = \dfrac{1}{2} [\dfrac{sen( n \pi)}{n \omega} - \dfrac{cos(n \dfrac{\pi}{2}) - 1}{(n \omega)^2}]}$\\
$sen (n \pi) = 0$\\
$\displaystyle{a_n = \dfrac{1}{2} [  \dfrac{1 - cos(n \omega)}{(n \omega)^2}]}$\\
Término $b_0$\\

$\displaystyle{b_n = \dfrac{2}{T} \int_{\delta}^{\delta + T} f(t) sen (n \omega t)dt = \dfrac{2}{4} \int_{-2}^{2} f(t) sen (n \omega t)dt = \dfrac{1}{2} ( + \int_{0}^{1} t sen (n \omega t) dt + \int_{1}^{2} sen (n \omega t) dt)}$\\
$\displaystyle{b_n = \dfrac{1}{2} [(\dfrac{- cos(n \omega t) t}{n \omega} + \int_{0}^{1} \dfrac{cos(n \omega t)}{n \omega} dt)  \mid_{0}^{1}  - \dfrac{cos(n \omega t) }{n \omega} \mid_{1}^{2}]}$\\
$\displaystyle{b_n = \dfrac{1}{2} [(\dfrac{- cos(n \omega t) t}{n \omega} + \dfrac{sen(n \omega t)}{(n \omega)^2})  \mid_{0}^{1}  - \dfrac{cos(n \omega t) }{n \omega} \mid_{1}^{2}]}$\\
$\displaystyle{b_n = \dfrac{1}{2} [(-\dfrac{cos(n \omega ) - 0cos(0)}{n \omega} + \dfrac{sen(n \omega)- sen(0)}{(n \omega)^2}) - \dfrac{cos(2 n \omega)- cos(n \omega) }{n \omega}]}$\\
$\displaystyle{b_n = \dfrac{1}{2} [\dfrac{sen(n \omega)}{(n \omega)^2}  - \dfrac{cos(2 n \omega)}{n \omega}]}$\\
$\displaystyle{b_n = \dfrac{1}{2} [\dfrac{sen(n \omega)}{(n \omega)^2}  - \dfrac{(-1)^n}{n \omega}]}$\\
Serie de Fourier \\
$\displaystyle{f(t) = \dfrac{1}{2} a_0 + \sum_{n = 1}^{\infty} (a_n cos(nt) + b_n sen(nt))}$\\
$\displaystyle{f(t) = \dfrac{3}{8} a_0 + \dfrac{1}{2} \sum_{n = 1}^{\infty}( (\dfrac{1 - cos(n \omega)}{(n \omega)^2}) cos(nt) + (\dfrac{sen(n \omega)}{(n \omega)^2}  - \dfrac{(-1)^n}{n \omega}) sen(nt))}$\\
\end{center}

\begin{center}
Transformada de Fourier\\
$\displaystyle{F(j \omega) = \int_{-\infty}^{\infty} f(t) e^{-j \omega t} dt}$\\
$\displaystyle{F(j \omega) = \int_{-2}^{2} f(t) e^{-j \omega t} dt}$\\
$\displaystyle{F(j \omega) = 0 + \int_{0}^{1} t e^{-j \omega t} dt + \int_{1}^{2} e^{-j \omega t} dt}$\\
$\displaystyle{F(j \omega) =(-\dfrac{ t e^{-j \omega t}}{j \omega} - \dfrac{e^{-j \omega t}}{(j \omega)^2}) \mid_{0}^ {1} - \dfrac{ e^{-j \omega t}}{j \omega} \mid_{1}^{2}}$\\
$\displaystyle{F(j \omega) =(-\dfrac{e^{-j \omega }-0}{j \omega} - \dfrac{e^{-j \omega} - 1}{(j \omega)^2}) - \dfrac{ e^{-2j \omega }- e^{-j \omega }}{j \omega} }$\\
$\displaystyle{F(j \omega) =  \dfrac{1-e^{-j \omega}}{(j \omega)^2}  - \dfrac{ e^{-2j \omega }}{j \omega}}$\\
Espectro de Frecuencias

\end{center}
\textbf{3) Defina:}
-Defina una funcion ortonormal\\
la expresion $\left( u,u \right) ={ \parallel u\parallel  }^{ 2 }$ se llama norma cuadrada.Por tanto podemos definir la norma cuadrada de una funcion como:\\
$
{ \parallel { \phi  }_{ n }(x)\parallel  }^{ 2 }=\int _{ a }^{ b }{ { \phi  }_{ n }^{ 2 }dx,\quad \quad  } \parallel { \phi  }_{ n }(x)\parallel =\sqrt { \int _{ a }^{ b }{ { \phi  }_{ n }^{ 2 }dx,\quad \quad  }  } \\ 
$
Si$\left\{ { \phi  }_{ n }(x) \right\}$ es un conjunto ortogonal en [a,b] con la propiedad de que $\parallel { \phi  }_{ n }(x)\parallel =1$,para todo n,entonces se llama conjunto ortonormal en [a,b]\\\\\\
-Dedusca los coeficientes de la S.F en terminos de seno de 1/2 recorrido.\\\\
Se tiene la forma general de la serie de Fourier:\\\\
$f(x)=\frac { { a }_{ 0 } }{ 2 } +\sum _{ n=1 }^{ \infty  }{ ({ a }_{ n }cos\frac { n\pi  }{ p } x+{ a }_{ n }sen\frac { n\pi  }{ p } x) } \\ $\\\\
Si multiplicamos la ecuacion 1 por .. e integramos y usamos los resultados .\\
\begin{center}
$\displaystyle{ \int _{ -p }^{ p }{ f(x){ a }_{ n }sen\frac { m\pi  }{ p } xdx } =\frac { { a }_{ 0 } }{ 2 } \int _{ -p }^{ p }{ sen\frac { m\pi  }{ p } x } dx+\sum _{ n=1 }^{ \infty  }{ ({ a }_{ n }\int _{ -p }^{ p }{ { a }_{ n }cos\frac { n\pi  }{ p } x*sen\frac { m\pi  }{ p } x } dx}}$\\
$\displaystyle{+{ b }_{ n }\int _{ -p }^{ p }{ sen\frac { m\pi  }{ p } x*sen\frac { m\pi  }{ p } x } dx) } $
\end{center}
Mediante ortogonalidad tenemos:\\\\
$\int _{ -p }^{ p }{ sen\frac { m\pi  }{ p }  } xdx=0\quad \quad \quad ,m>0,\int _{ -p }^{ p }{ sen\frac { m\pi  }{ p }  } x*sen\frac { n\pi  }{ p } xdx=0\\\\
\int _{ -p }^{ p }{ sen\frac { m\pi  }{ p }  } x*sen\frac { n\pi  }{ p } xdx= \left\{ {0 ;  m \neq n \atop p ; m=n} \right.\\\\
encontramos\quad que\quad :\\ b_n=\int _{ -p }^{ p }{ f(x)*sen\frac { n\pi  }{ p }  } xdx$\\
\textbf{4) Defina y explique que es una Serie de Bessel}
Es un caso particular de la series Generalizadas de Fourier, basado en la Función de Bessel que es la función: \\
${x^2 \frac{d^2 y}{dx^2} + x \frac{dy}{dx} + (x^2 - \alpha^2)y = 0}$\\
Estas series son usadas en la solución de las ecuaciones diferenciales parciales particularmente en los sistemas de coordendas cilíndricas.\\

La serie de Fourier-Bessel de una función $f$ definida en el intervalo (0,b) está dada por: \\
\begin{center}
 $\displaystyle{f(x) = \sum_{i = 1}^{\infty} c_i J_n (\alpha_i x)}$ \\
 $\displaystyle{c_i = \dfrac{2}{b^2 J^2_{n+1} (\alpha_i b)} \int_{0}^{b} x J_n (\alpha _i x) f(x)}$\\
 donde los $\alpha_i$ están definidos por $J_n (\alpha b) = 0$\\
 \end{center} 
\end{document}