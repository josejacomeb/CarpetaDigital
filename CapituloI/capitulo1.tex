


\chapter{variable compleja}%\label{cap.introduccion}
\section{Variable compleja}
\begin{small}
Los números complejos se dice $z$ que se puede definir como pares ordenados $$z=\langle x,y\rangle$$ 
de números reales $x$ e $y$ con las operaciones de suma y producto. Se suele identificar los pares $\langle x,0 \rangle$ con los números reales $x$. El conjunto de los números contiene a los números reales como subconjunto. \\
Los números complejos de la forma $\langle 0,y\rangle$ se llama números imaginarios puros. Los números reales $x$ e $y$ en la expresión se conocen respectivamente, como parte real y parte imaginaria de $z$. $$Re\langle z\rangle=x \qquad Im\langle z\rangle=y$$ 
Dos números complejos $\langle x_{1},y_{1} \rangle $ y $\langle x_{2},y_{2} \rangle $ se dicen iguales si tienen iguales la parte real e imaginarios. Es decir $$\langle x_{1},y_{1} \rangle = \langle x_{2},y_{2} \rangle $$ si y solo si $$ x_{1}=x_{2} \wedge y_{1}=y_{2} $$ La suma $ z_{1} + z_{2} $ y el producto $ z_{1} z_{2}$ de dos números complejos $ z_{1}= \langle x_{1},y_{1} \rangle $ y $ z_{2}= \langle x_{2},y_{2} \rangle $ se definen por las ecuaciones $$\langle x_{1},y_{1} \rangle + \langle x_{2},y_{2} \rangle = \langle x_{1} + x_{2}, y_{1} + y_{2} \rangle $$
$$ \langle x_{1},y_{1} \rangle \langle x_{2},y_{2} \rangle = \langle x_{1} x_{2} - y_{1} y_{2}, y_{1} x_{2} - x_{1} y_{2} \rangle $$
En particular $ \langle x, 0 \rangle + \langle 0, y \rangle = \langle x, y \rangle$ y $ \langle 0, y \rangle \langle y, 0 \rangle = \langle 0, y \rangle $ luego $$ \langle x, y \rangle = \langle x, 0 \rangle + \langle 0, 1 \rangle \langle y, 0 \rangle $$
El sistema de los números complejos es un consecuencia una extensión natural de los números reales.\\
Pensando en un número real como $x$ o como $ \langle x,0 \rangle $ y denotamos por $i$ al numero imaginario puro $ \langle 0, 1 \rangle $ podemos ver $$ \langle x, y \rangle = x + y\textit{i} $$
Asimismo con el convenio $ z^{2}= z z, z^{3} = z z^{2} etc...$ Hallamos $$ t^{2} = \langle 0, 1 \rangle \langle 0, 1 \rangle = \langle -1, 0 \rangle $$
Es decir $$ t^{2} = -1 $$ 
Se puede divisar la expresión $$ \langle x_{1} + y_{1}\textit{i} \rangle + \langle x_{2} + y_{2}\textit{i} \rangle = \langle x_{1} + x_{2} \rangle + \langle y_{1} + y_{2}\rangle \textit{i}$$ $$ \langle x_{1} + y_{1}\textit{i} \rangle \langle x_{2} + y_{2}\textit{i} \rangle = \langle x_{1} x_{2} - y_{1} y_{2} \rangle + \langle y_{1} x_{2} + x_{1} y_{2} \rangle \textit{i} $$
Observese que los miembros de la derecha en esas ecuaciones se pueden obtener formalmente manipulando los términos de la izquierda como se sólo contuvieron números numeros reales y sustituyendo como se sólo contuvieron números reales y sustituyendo $t^{2} $ por $-1$ cuando aparezca

\end{small}

\section{Propiedades}
\begin{small}

\end{small}

%\chapter*{limites y con. de func. complejas}\label{cap.introduccion}
afafa fafaf afafa 

%\subsection{subsección1}
%Ble ble ble
%\subsubsection{subsubsección1}
%Bli bli bli
%\paragraph{párrafo1}
%Blo blo blo

