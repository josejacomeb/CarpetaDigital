\documentclass[10pt,a4paper]{article}
\usepackage[utf8]{inputenc}
\usepackage{amsmath}
\usepackage{amsfonts}
\usepackage{amssymb}
\usepackage{graphicx}
\usepackage{fourier}
\author{José Jácome}
\title{Capítulo VII}
\begin{document}
%\chapter{Capitulo VII}
\section{SERIES DE POTENCIAS, DE TAYLOR Y DE LAURENT}
\subsection{Series de Potencias}
Una serie de potencias en el plano complejo es de la forma siguiente:
\begin{equation}
\sum_{n = 0}^{\infty} c_n (z-z_0)^n = c_0 + c_1(z-z_0) + c_2 (z-z_0)^2 + ... +  c_n (z-z_0)^n + ...
\end{equation}
donde $c_n$ son constantes reales y complejos llamados coeficientes "$z_0$" es constante y se llama \textit{centro de la serie},"$z$" es la variable compleja. \\
Si $z_0 = 0$ , la serie (1) se reduce a la forma $\displaystyle{\sum_{n = 0}^{\infty} c_n z^n = c_0 + c_1 z + c_2 z^2 + ... +  c_n z^n + ...}$
\end{document}